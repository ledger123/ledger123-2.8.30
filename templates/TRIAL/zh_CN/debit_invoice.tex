%% First Choice Internet
%% TRIAL - Debit Invoice, Chinese simplified
%% B.Plagge 2008-07-05
<%include settings.tex%>

\begin{document}
\begin{CJK}{UTF8}{arialuni}

\vspace*{-3.3cm}
<%include letterhead.tex%>
\vspace*{1.5cm}

\parbox[t]{.55\textwidth}{
\textbf{送货到:} \\
\vspace{0.3cm}
<%include shipto_cn.tex%>
}
\parbox[t]{.45\textwidth}{
\vspace{0.3cm}
<%include vendor_cn.tex%>
}

\hfill \\
-
\vspace{1.0cm}

\centerline{\textbf{借 记 发 票}}

\vspace{1cm}

\begin{tabularx}{\textwidth}{*{8}{|X}|} \hline
  \textbf{发票号码} & \textbf{日期} & \textbf{限期} & \textbf{订单号码}
  & \textbf{制单员}
  <%if shippingpoint%>
  & \textbf{装送点}
  <%end shippingpoint>
  <%if ponumber%>
  & \textbf{订单号码}
  <%end ponumber%>
  <%if shipvia%>
  & \textbf{经由}
  <%end shipvia%>
  \\ [0.5em]
  \hline
  <%invnumber%> & <%invdate%> & <%duedate%> & <%ordnumber%> & <%employee%>
  <%if shippingpoint%>
  & <%shippingpoint%>
  <%end shippingpoint>
  <%if ponumber%>
  & <%ponumber%>
  <%end ponumber%>
  <%if shipvia%>
  & <%shipvia%>
  <%end shipvia%>
  \\
  \hline
\end{tabularx}

\vspace{1cm}
发票货币: \textbf{<%currency%>}.
\vspace{0.5cm}

\begin{longtable}{|ll p{6cm} @{\extracolsep\fill} lrlrrr|} \hline
\xstrut
  \textbf{项目} & \textbf{原料编号} & \textbf{说明} & \textbf{发货日期} &
	\textbf{数量} & \textbf{単位}  & \textbf{单价} & \textbf{折扣%} & \textbf{金额} \\     
  \hline
\endfirsthead
  \multicolumn{8}{l}{\emph{继续上页 ...}} \\
  \hline
  \textbf{项目} & \textbf{原料编号} & \textbf{说明} & \textbf{发货日期} &
	\textbf{数量} & \textbf{単位}  & \textbf{单价} & \textbf{折扣%} & \textbf{金额} \\     
  \hline
\endhead
   \hline \multicolumn{8}{r}{\emph{下页继续 ...}}
\endfoot
   \hline
	<%if taxincluded%>
	   \multicolumn{8}{|lr} \textbf{小计(无税)} & <%subtotal%> \\
	<%end taxincluded%>
	<%if not taxincluded%>
	   \multicolumn{8}{|lr} \textbf{小计} & <%subtotal%> \\
	<%end taxincluded%>

   <%foreach tax%>
     \multicolumn{8}{|lr} \textbf{(增值税} & <%tax%>\\
   <%end tax%>  
   <%if paid%>
     \multicolumn{8}{|lr} \textbf{{已收金额}  & - <%paid%> \\
   <%end paid%>
   <%if total%>
     \hline
     \multicolumn{8}{|rr} \textbf{未付} & <%total%> \\
   <%end total%>
  \hline
\endlastfoot
<%foreach number%>
  <%runningnumber%> & <%number%> & <%description%> & <%deliverydate%> & 
  <%qty%> & <%unit%> & <%sellprice%> & <%discountrate%> & <%linetotal%> \\
  <%if itemnotes%>& & <%itemnotes%> & & & & & &\\<%end itemnotes%>
<%end number%>
\end{longtable}

\parbox{\textwidth}{

\vspace{0.2cm}

<%notes%>
}

\vfill

<%if paid_1%>
\begin{tabularx}{10cm}{@{}lXlr@{}}
  \multicolumn{4}{c}{\textbf{付款}} \\
  \hline
  \textbf{日期} & & \textbf{付款方式} & \textbf{金额} \\
<%end paid_1%>
<%foreach payment%>
  <%paymentdate%> & <%paymentaccount%> & <%paymentsource%> & <%payment%> \\
<%end payment%>
<%if paid_1%>
\end{tabularx}
<%end paid_1%>

\end{CJK}  
\end{document}

